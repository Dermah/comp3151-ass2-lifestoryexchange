\documentclass[a4paper]{article}
\title{LSE}
\author{Steven Antolovich (sant964) \and Wing Kam Chan (wkch250)}

\usepackage{new3151defs}
\usepackage{listings}
\lstset{language=Promela}
\usepackage{tikz}
\usetikzlibrary{automata,positioning}

\begin{document}

\maketitle

\section{Assumptions}
\begin{quote}
If $a$ announces to enter a moment of contemplative vegetation, then no compatible senior will do the same.
\end{quote}
We assume this to mean:
\begin{quote}
If $a$ announces to enter a moment of contemplative vegetation, then no compatible senior that is not otherwise dead or already engaged in an LSE will do the same.
\end{quote}

\begin{quote}
For simplicity, we may assume that no senior dies between exchanging a message and making an announcement... The actual death will then occur according to an even distribution before sending a message
\end{quote}
We assume this to mean that no senior dies between exchanging their first message and making an announcement. If it were possible to die between every message, it would be impossible to calculate an even distribution since the number of messages the senior will send before announcing is unknown.

\end{document}
