\documentclass[a4paper]{article}
\title{LSE}
\author{Steven Antolovich (sant964) \and Wing Kam Chan (wkch250)}

\usepackage{new3151defs}
\usepackage{listings}
\lstset{language=Promela}
\usepackage{tikz}
\usetikzlibrary{automata,positioning}

\begin{document}

\maketitle

\section{Assumptions}
\begin{quote}
If $a$ announces to enter a moment of contemplative vegetation, then no compatible senior will do the same.
\end{quote}
We assume this to mean:
\begin{quote}
If $a$ announces to enter a moment of contemplative vegetation, then no compatible senior that is not otherwise dead or already engaged in an LSE will do the same.
\end{quote}

\begin{quote}
For simplicity, we may assume that no senior dies between exchanging a message and making an announcement... The actual death will then occur according to an even distribution before sending a message
\end{quote}
We assume this to mean that no senior dies between exchanging their first message and making an announcement. If it were possible to die between every message, it would be impossible to calculate an even distribution since the number of messages the senior will send before announcing is unknown.

\section{Analysis}

\section{Verification}

\section{Comparison}


\section{Appendix}
\subsection{Twenty Senior Compatibility Matrix File}
\begin{verbatim}
20
01000100100100000100
10110001100100000011
01011010001100010010
01100000011100100000
00100010011110101001
10000010011110100000
00101100001100110000
01000000001100100001
11000000011101100100
00011100100100000100
00111111100100010000
11111111111000000001
00001100000000000101
00000000100000010001
00011111100000000001
00100010001001000001
00001000000000000001
10000000110010000000
01100000000000000000
01001001000111111000
3 7
14
\end{verbatim}

\end{document}
