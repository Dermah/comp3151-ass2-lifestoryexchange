\documentclass[a4paper]{article}
\title{Life Story Exchange}
\author{Steven Antolovich (sant964) \and Wing Kam Chan (wkch250)}

\usepackage{new3151defs}
\usepackage{listings}
\lstset{language=Promela}
\usepackage{tikz}
\usetikzlibrary{automata,positioning}

\begin{document}

\maketitle

\section{Assumptions}
\begin{quote}
If $a$ announces to enter a moment of contemplative vegetation, then no compatible senior will do the same.
\end{quote}
We assume this to mean:
\begin{quote}
If $a$ announces to enter a moment of contemplative vegetation, then no compatible senior that is not otherwise dead or already engaged in an LSE will do the same.
\end{quote}

\begin{quote}
For simplicity, we may assume that no senior dies between exchanging a message and making an announcement... The actual death will then occur according to an even distribution before sending a message
\end{quote}
We assume this to mean that no senior dies between exchanging their first message and making an announcement. If it were possible to die between every message, it would be impossible to calculate an even distribution since the number of messages the senior will send before announcing is unknown.

\section{Analysis}
\subsection{Message Complexity}
The algorithm uses a fairly simple set of messages to successfully negotiate an LSE. There are only three types of message: a request to LSE with someone (\texttt{LSE\_I\_WANT\_TO\_EXCHANGE}), an acknowledgement of this request (\texttt{LSE\_THAT\_SOUNDS\_GREAT}) or a rejection of the request (\texttt{LSE\_NO\_THANKS}). 

When a Senior 1 sends a request to Senior 2, it waits for a response from Senior 2 only. Attempts to initiate an exchange with Senior 1 by anyone else will be met with a \texttt{LSE\_NO\_THANKS}. Senior 1 can receive from Senior 2 either a \texttt{LSE\_THAT\_SOUNDS\_GREAT} or a \texttt{LSE\_I\_WANT\_TO\_EXCHANGE} as confirmation. In the first case, Senior 2 has directly acknowledged and announced, while in the second case, both Senior 2 will eventually receive Senior 1's initial request. The two discover that they are requesting each other, resulting in a successful negotiation. 

If a senior is not waiting to hear back from someone and they receive a request, they are free to immediately acknowledge and announce the request. 

If a senior is rejected by the senior it is waiting to hear from, it makes a note to try and contact them again later and attempts to request someone else. 

If a senior is dead, they do not reply to any messages. Surrounding seniors notice this and stop trying to contact the dead senior. 

A senior that has announced cannot be used by any seniors, effectively making them dead to any compatible seniors. In my implementation, a senior that has announced does not respond to messages, meaning surrounding seniors stop trying to contact them because they think they are dead. This allows death and success to be combined and thus handled in the same way. 

\subsection{Time Complexity}
This algorithm is not efficient.



\section{Verification}
\subsection{Promela}

\section{Comparison}


\section{Appendix}
\subsection{Twenty Senior Compatibility Matrix File}
\begin{verbatim}
20
01000100100100000100
10110001100100000011
01011010001100010010
01100000011100100000
00100010011110101001
10000010011110100000
00101100001100110000
01000000001100100001
11000000011101100100
00011100100100000100
00111111100100010000
11111111111000000001
00001100000000000101
00000000100000010001
00011111100000000001
00100010001001000001
00001000000000000001
10000000110010000000
01100000000000000000
01001001000111111000
3 7
14
\end{verbatim}

\end{document}
